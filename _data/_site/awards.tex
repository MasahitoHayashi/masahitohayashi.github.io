
\section{Honors, Awards, \& Fellowships}
\begin{enumerate}
\item \textbf{\textit{2017 IEEE Fellow:}}  
For contributions to Shannon theory, information-theoretic security, and quantum information theory.
 


\item \textbf{\textit{12th Japan Academy Medal:}}  
``Information Theory and Quantum Information Theory for Finite-Coding-Length''
This prize is awarded by The Japan Academy to up to 6 prestigious Japanese researchers under 45 years old among all scientific areas.
Awardees are selected from among the annual recipients of the JSPS Prize.


\item \textbf{\textit{FY2015 JSPS PRIZE:}}  
``Information Theory and Quantum Information Theory for Finite-Coding-Length''
The JSPS PRIZE is meant to recognize at an early stage in their careers young researchers with fresh ideas who have the potential to become world leaders in their fields, while helping to enhance their opportunities to advance their research and make breakthroughs.
Twenty-five researchers under 45 years old were selected for this year's PRIZE. Their fields of research run the spectrum from the humanities and social sciences to the natural sciences.



\item \textbf{\textit{2011 IEEE Information Theory Society Paper Award:}} ``Information spectrum approach to second-order coding rate in channel coding''
{\em IEEE Transactions on Information Theory},
Vol. 55, No. 11, 4947--4966 (2009). 
This prize is the most distinguished paper award in the information theory community.

\item \textit{Senior Member of IEEE}, 2013. 

\item \textit{Japan IBM Prize in the Computer Science Section 2010:} ``Universal protocol in quantum information and its application to quantum key distribution"
This prize is one of the most distinguished prizes in Japan in information science for researchers under 45 across Japan.

\item \textit{Funai Foundation for Information Technology Award in the Computer Science Category 2010:} 
``Universal quantum information protocol and its application to quantum cryptography''.

\item \textit{Research Fellowship for Young Scientists}, 
the Japan Society for the Promotion of Science,
1998.

\item \textit{2001 SITA Encouragement Award} (by The Society of Information Theory and its Applications (SITA)):
``Variable length universal entanglement concentration by local operations''.

\item \textit{16th TEPIA Video Award, 2006}
(by Association for Technological Excellence Promoting Innovative Advances (TEPIA)):
This prize is awarded to 
Public Video for ERATO Quantum computation and information project.
I managed the production process as Research Manager.

%\item Summer Research Fellowship, Duke University, 2007, 2008.
%\item Institute on Computational Economics Fellow, University of Chicago, 2006.
%\item Outstanding Teaching Assistant Award, Duke University, 2006.
%\item Undergraduate Research Award, North Carolina State University, 2004.
% \item Omicron Delta Epsilon, International Economics Honor Society, 2004.
%\item Phi Beta Kappa, Zeta of North Carolina, 2003.
% \item Pi Mu Epsilon, National Mathematics Honor Society, 2003.
%\item NSF Research Experiences for Undergraduates, Colorado School of Mines, 2002.
% \item Phi Kappa Phi, 2002.
% \item Golden Key International Honour Society, 2002.
% \item Dean's List, North Carolina State University, 2000--2004.
\end{enumerate}

\section{Professional Activities}

\subsection{Editorial Board Member}
\begin{enumerate}
\item Editorial Board Member, International Journal of Quantum Information (IJQI) World Scientific, 2002 -- Current.
\item Advisory Board Member, International Journal On Advances in Security, IARIA, 2009 -- Current. (Editorial Board Member: 2008-2009)
\end{enumerate}

\subsection{Organization of International Meetings}
\begin{enumerate}
\setcounter{enumi}{2}
\item Organizing Committee Chair:
{\em ERATO Conference on Quantum Information Science 2003 (EQIS 03)}, 
Doshiha Univ., Kyoto, Japan, September 4-6, 2003.

\item Organizing Committee and Program Committee:
{\em COE Symposium on Quantum Information Theory}, 
Doshiha Univ., Kyoto, Japan, September 2-3, 2003.

\item Program Committee Vice-Chair: 
{\em ERATO Conference on Quantum Information Science 2004 (EQIS 04)},
Hitotsubashi-memorial hall, Tokyo, Japan, September 1--5, 2004.

\item Program Committee Vice-Chair: 
{\em ERATO Conference on Quantum Information Science 2005 (EQIS 05)}, 
JST-Museum Hall, Tokyo, Japan, August 26--30, 2005.
\item Organizing Committee Chair: 
{\em COE-Kakenhi Workshop on Quantum Information Theory and Quantum Statistical Inference}, 
University of Tokyo, Hongo, Tokyo,
November 17--18, (2005)
\item Local Committee: 
{\em The 8th International Conference on Quantum Communication, Measurement, and Computing (QCMC2006)}, 
Tsukuba Epocal, Tsukuba, Japan, 
November 28-- Decmeber 3, (2006).
\item Program Committee: 
{\em Asian Conference on Quantum Information Science 2006 (AQIS06)}, 
Beijing, China, September 1--4, 2006.
\item Technical Program Committee, 
{\em The First International Workshop on Quantum Security (QSEC2007)}, 
Guadeloupe, French Caribbean, France, January 2--6, 2007. Organized by IEEE France and IARIA.
\item Organizing Committee Chair: 
{\em Special seminar series on quantum information}, 
National Institute of Informatics (NII), Tokyo, February 1 -- March 21, 2007.

\item Program Committee: 
{\em The 3rd Workshop on Theory of Quantum Computation, 
Communication, and Cryptography (TQC2008)}, 
The University of Tokyo, Tokyo Japan, January 30 -- February 1, 2008.
\item Technical Program Committee Co-Chair: 
{\em ICQNM 2008, The Second International Conference on Quantum, Nano, and Micro Technologies}, 
Sainte Luce, Martinique, France, February 10--15, 2008. Organized by IEEE France and IARIA.
\item Main Organizer: 
{\em GSIS \& DEX-SMI Workshop on Quantum statistical inference and entanglement Graduate School of Information Sciences}, 
Tohoku University, 
February, 11--12, 2008
\item International Advisory Committee: 
{\em International School and Conference on Quantum Information (ISCQI - 2008)},
Institute of Physics (IOP), Bhubaneswar, Orissa, India, March 4--12, 2008
\item Program Committee: 
{\em Asian Conference on Quantum Information Science 2008 (AQIS08)}, 
Souel, Korea, 
August 25--26, 2008.

\item Organizing Committee: 
{\em International Workshop on Statistical-Mechanical Informatics 2008 (IW-SMI2008)}, Sendai, Japan, September 14--17, 2008.

\item Organizing Committee Chair: 
{\em GSIS Workshop on Quantum Information Theory}, 
Sendai, Japan, November 5--7, 2008

\item Organizing Committee: 
{\em Multicritical Behaviour of Spin Glasses and Quantum Error Correcting Codes (MBQEC)}, 
Centennial Hall, Ookayama campus, Tokyo Institute of Technology, Tokyo, Japan, November 17--19 (2008)

\item Organizing committee chair: 
{\em DEX-SMI workshop on quantum statistical inference}, 
National Institute of Informatics (NII), Tokyo, Japan
March, 2--4 (2009).

\item Program Committee: 
{\em ICQNM 2009, The Third International Conference on Quantum, Nano and Micro Technologies}, 
Cancun, Mexico, February 1--6 (2009)
\item Program Committee: 
{\em Workshop on Theory of Quantum Computation, Communication, and Cryptography (TQC2009)}, Waterloo, Canada, May 11--13 (2009)
\item Program Committee: 
{\em ICQNM 2010, The Fourth International Conference on Quantum, Nano and Micro Technologies}, 
St. Maarten, Netherlands Antilles, February 10--16 (2010).
\item Program Committee (Special Area Chair): 
{\em ICQNM 2011, The Fifth International Conference on Quantum, Nano and Micro Technologies}, 
Nice, France, August 21--27 (2011).
\item Technical Program Committee: 
{\em ICQNM 2012, The Sixth International Conference on Quantum, Nano and Micro Technologies}, 
Rome, Italy, August 19--24 (2012)

\item Program Committee: {\em 12th Asian Quantum Information Science conference (AQIS12)}, 
School of Physical Science and Technology, Soochow University, Suzhou, China, August 23--26 (2012).

\item Workshop Co-chair:
{\em Japan-Singapore Workshop on Multi-user Quantum Networks}, 
Centre for Quantum Technologies, National University of Singapore, Singapore, 
September 17 -- 20 (2012)

\item
Special Area Chair: 
{\em ICQNM 2013, The Seventh International Conference on Quantum, Nano and Micro Technologies}, 
Barcelona, Spain, August 25 -- 31 (2013)

\item
Program Committee: 
{\em 13th Asian Quantum Information Science conference (AQIS 13)}, 
IMSC Chennai, India, August 25--30 (2013)

\item
Program Committee: 
{\em The 7th International Conference on Information Theoretic Security (ICITS)}, NTU, Singapore, November, 28--30 (2013)

\item
Program Committee: 
{\em 2014 International Symposium on Information Theory (ISIT)}, 
Honolulu, Hawaii, June 29 -- July 4 (2014)

\item
Program Committee: 
{\em 14th Asian Quantum Information Science conference (AQIS 14)}, 
Kyoto, Japan, August 21--25  (2014)

\item
Special Area Chair: 
{\em ICQNM 2014, The Eighth International Conference on Quantum, Nano and Micro Technologies}, 
Lisbon, Portugal,
November 16 -- 20 (2014)

\item
Program Committee: 
{\em The 9th Conference on the Theory of Quantum Computation, Communication and
Cryptography (TQC2014)}, 
National University of Singapore, Singapore,
May, 21-23 (2014)

\item
Organizer:
{\em Australia-Japan Workshop on Multi-user Quantum Networks},
University of Technology, Sydney,
October, 22--24 (2014)


\item
Program Committee: 
{\em Quantum Information Processing (QIP 2015)}, 
Sydney January (2015)

\item
Program Committee: 
{\em 2015 International Symposium on Information Theory (ISIT)}, 
Hong Kong, June 14--19  (2015)

\item
Special Area Chair: 
{\em ICQNM 2015, The Seventh International Conference on Quantum, Nano and Micro Technologies,} Venice, Italy, August 23 - 28, 2015.

\item
Program Committee: 
{\em 2015 IEEE Information Theory Workshop (ITW),} 
Jeju Island, Koreag, October 11-15, 2015.

\item
Special Area Chair: 
{\em ICQNM 2016, The Tenth International Conference on Quantum, Nano and Micro Technologies}, Nice, France, July 24 - 28, 2016.

\item
Program Committee:
{\em 16th Asian Quantum Information Science conference (AQIS'16)}, Academia Sinica, Taipei, Taiwan, August 29-September 1, 2016.

\item
Conference Co-Organizer:
{\em Beyond I.I.D. in Information Theory 2017}, 
Institute for Mathematical Sciences, National University of Singapore, Singapore, July 24-28, 2017

\item
Steering Committee: 
{\em ICQNM 2017, The Eleventh International Conference on Quantum, Nano and Micro Technologies}, 
Rome, Italy, September 10 - 14, 2017.

\item
Program Committee:
{\em 10th International Conference on Information Theoretic Security (ICITS)}, 
Hong Kong, China, on November 29 - December 2, 2017.
 
\item
Program Committee: 
{\em Quantum Information Processing 2018 (QIP)}, 
Aula Congress Centre, the TU Delft, The Netherlands,
January 15 - 19, 2018.

\item
Program Committee: 
{\em 2018 International Symposium on Information Theory (ISIT)}, 
Vail Cascade, CO, USA, June 17 -- 22,  2018.

\item
Program Committee: 
{\em The 13th Conference on the Theory of Quantum Computation, Communication and Cryptography (TQC2018)}, 
University of Technology, Sydney,
July, 16 -- 18, 2018.

\item
Steering Committee: 
{\em ICQNM 2018, The Twelfth International Conference on Quantum, Nano and Micro Technologies}, 
Venice, Italy, September 16 -- 20, 2018.

%\item
%Program Committee: 
%{\em Trustworthy Quantum Information}, Singapore, 2018.

\item
Organizer:
Nagoya-SUSTech Quantum Information Workshop,
Nagoya, Japan, April 11 -- 13, 2019.

\item
Program Committee: 
{\em 2019 International Symposium on Information Theory (ISIT)}, 
Paris, France, July, 7 -- 12, 2019. 
%Vail Cascade, CO, USA, June 17 - 22,  2018.

\item
Steering Committee: 
{\em ICQNM 2019, The Thirteenth International Conference on Quantum, Nano and Micro Technologies}, 
Nice, France, October 27 -- 31, 2019.

\item
Organizing Committee: 
{\em Quantum Information Processing in Non-Markovian Quantum Complex Systems}, 
Nagoya, Japan, December  9 -- 12, 2019. 

\item
Local Organizing Committee: 
{\em Quantum Information Processing 2020 (QIP)}, 
%{\em 23rd Annual Conference on Quantum Information Processing}, 
Shenzhen, China, January 6 -- 10, 2020.

\end{enumerate}

\subsection{Review Committee for PhD Thesis}
\begin{enumerate}
\setcounter{enumi}{51}
\item 2005, Review Committee Member for Two Ph.D. Candidates, Graduate School of Information Science and Technology, The University of Tokyo, Japan.
\item 2006, Review Committee Member for a Ph.D. Candidate, Graduate School of Information Science and Technology, The University of Tokyo, Japan.
\item 2008, External examiner (referent), Faculty of Mathematics and Natural Sciences, University of Leiden, Netherlands
\item 2008, External examiner, National Institute of Education, Nanyang Technological University, Singapore.
\item 2008, External examiner, Department of Communication and Integrated Systems, Graduate School of Science and Engineering, Tokyo Institute of Technology.
\item 2012, External examiner, School of Computer Science, McGill University, Canada.
\item 2013, External examiner, Graduate School of Information Sciences, Tohoku University, Japan.

\item 2014, External examiner, 
Department of Electrical and Computer Engineering,
National University of Singapore, Singapore.

\item 2016, Review Committee Member for a Ph.D. Candidate, Graduate School of Mathematics, Nagoya University, Japan.

\item 2017, Review Committee Member for a Ph.D. Candidate, Graduate School of Mathematics, Nagoya University, Japan.

%\item 2018, External examiner, 
%Department of Computer Science, National University of Singapore, Singapore.

\end{enumerate}

\subsection{Other Kinds of Review Committees}
\begin{enumerate}
\setcounter{enumi}{61}
\item 2017,
External Reviewer of the Computer Science Department and 
the Physics Department at the University of California at Santa Barbara
(Promotion to Full Professor)

\item 2019, External Reviewer of 
the main Chilean funding agency, National Fund for Scientific and Technological Development (FONDECYT)

\item 2018, External Reviewer of 
the French National Research Agency (ANR), France

\item 2018, External Reviewer of 
the Israel Science Foundation (ISF), Israel

\item 2017, External Reviewer of 
the executive government agency of National Science Centre, Poland

\item 2015, 2017 External Reviewer of 
Research Committee, the Czech Science Foundation.

\item 2014, External Reviewer of 
Research Committee, University of Macau (UM) for the 2014 Multi-Year Research Grant (MYRG).

\item 2012,
External Reviewer of 
TATA INSTITUTE OF FUNDAMENTAL RESEARCH.
(Promotion to Tenured Associate Professor)


\end{enumerate}

\subsection{Other professional activities}
\begin{enumerate}
\setcounter{enumi}{68}
\item I am one of founders of the
{\em Asian Conference on Quantum Information Science (AQIS)} conference series, which is a major international conference series on quantum information and computation.

%\item As a co-chair, I organised {\em Japan-Singapore Workshop on Multi-user Quantum Networks}, whose speakers include many international outstanding researchers. % as well as Japanese and Singaporian outstanding researchers.

%\item I have worked as a member of Organising/Program Committee (including Organising Committee Chair and Program Committee Vice-Chair) of several important conference series including (AQIS, TQC, QCMC, ICITS, and ICQNM).
\item As Research Manager, I oversaw the ERATO Quantum Computation and Information Project, 
which at the time was the largest research group for quantum information and computation in Japan. 

\item I have served as a referee for 
renowned international journals 
including the following journals

(journals related Nature)
\textit{Nature Photonics}, 
\textit{Nature Communications}, 
\textit{Nature Partner Journal Quantum Information}, 
\textit{Scientific Reports},

(Physics) \textit{Physical Review X}, 
\textit{Physical Review Letters}, 
\textit{Physical Review A}, 
\textit{Physical Letters A}, 
\textit{Journal of Physics A},
\textit{New Journal of Physics},
\textit{Annals of Physics},
\textit{Journal of Statistical Physics},

(Mathematical Physics)
\textit{Communications in Mathematical Physics}, 
\textit{Journal of Mathematical Physics},
\textit{Reports on Mathematical Physics},
\textit{Reviews in Mathematical Physics},

(IEEE)
\textit{Proceedings of the IEEE},
\textit{IEEE Transactions on Information Theory}, 
\textit{IEEE Transactions on Information Forensics \& Security},
\textit{IEEE Transactions on Wireless Communications},
\textit{IEEE Communications Letters},
\textit{IEEE Transactions on Communications}.

(Other areas)
\textit{Annals of Statistics},
\textit{Neurocomputing},
\textit{Information Geometry}.

\end{enumerate}


\end{document}


\section{Preprints}
\begin{enumerate}


\bibitem{Hp15}
S. Watanabe and  \textbf{M. Hayashi},
``Finite-length Analysis on Tail probability and Simple Hypothesis Testing for Markov Chain,''
arXiv:1401.3801
(Available online: http://arxiv.org/abs/1401.3801).


\bibitem{Hp10}
\textbf{M. Hayashi}, and T. Tsurumaru,
``More Efficient Privacy Amplification with Non-Uniform Random Seeds via Dual Universal Hash Function,''
(Available online: http://arxiv.org/abs/1311.5322).

\bibitem{Hp12}
\textbf{M. Hayashi},
``Optimal decoy intensity for decoy quantum key distribution,''
(Available online: http://arxiv.org/abs/1311.3003).
 
\bibitem{Hp2}
W. Kumagai, and \textbf{M. Hayashi},
``Second Order Asymptotics of Optimal Approximate Conversion for Probability Distributions and Entangled States and Its Application to LOCC Cloning,''
arXiv:1306.4166, 2013
(Available online: http://arxiv.org/abs/1306.4166).

\bibitem{Hp4}
\textbf{M. Hayashi},
``Fourier Analytic Approach to Quantum Estimation of Group Action,''
arXiv:1209.3463, 2012
(Available online: http://arxiv.org/abs/1209.3463).


\bibitem{Hp6}
\textbf{M. Hayashi}, 
``Quantum wiretap channel with non-uniform random number and its exponent and equivocation rate of leaked information,''
arXiv:1202.0325, 2012.
(Available online: http://arxiv.org/abs/1202.0325).

\bibitem{Hp8}
R. Matsumoto and \textbf{M. Hayashi},
``Universal strongly secure network coding with dependent and non-uniform messages,''
arXiv:1111.4174, 2011.
(Available online: http://arxiv.org/abs/1111.4174).


\bibitem{Hp9}
M. Owari and \textbf{M. Hayashi}, 
``Local hypothesis testing between a pure bipartite state and the white noise state,''
arXiv:1006.2744, 2010.
(Available online: http://arxiv.org/abs/1006.2744).


\end{enumerate}


%\bibitem[\textbf{Patents}]{Patents}
\newpage
\textbf{Ten career-best publications}

%\textbf{\color{red}TO BE EXTENDED!!!}
%Please attach a PDF of your recent significant publications (5 pages maximum). Refer to the Instructions to Applicants for further information. (This %question must be answered)

%i. provide the full reference for each of your ten best publications;
%ii. include any information relating to whether or not the publication was produced through an ARC funded Project/Fellowship on which you were a %Chief/Partner Investigator or Fellow;
%iii. add a statement of a maximum of 30 words explaining and justifying the impact or significance of each publication; and
%iv. asterisk any of the publications relevant to this Proposal.


\renewcommand{\refname}{}
\vspace{-20 mm}
\begin{thebibliography}{99}



%\item This book includes many advanced topics in quantum information such as quantum teleportation, superdense coding, quantum state transmission (quantum error-correction), quantum entanglement, quantum date compression, quantum information geometry, quantum state estimation, quantum state discrimination, and quantum cryptography.

%\item Unlike earlier treatments, the textbook requires knowledge of only linear algebra, probability theory, and quantum mechanics, while it treats the topics of quantum hypothesis testing and the discrimination of quantum states, and quantum channel coding (message transmission) with the minimal amount of math needed to convey their essence. 

%\item This book contains more than 240 exercises those provide readers practice that not only enriches their knowledge of quantum information theory, but also can equip them with the techniques necessary for pursuing their own research in this field.

\bibitem[\textbf{Award paper}]{1}

\bibitem{Ha09b} \textbf{M. Hayashi}, 
``Information spectrum approach to second-order coding rate in channel coding,''
{\em IEEE Transactions on Information Theory},
Vol. 55, No. 11, 4947--4966 (2009). 
(Citation: \textbf{\textit{79}}.  This paper was awarded the
\textbf{\em 2011 IEEE Information Theory Society Paper Award}, which is
the most distinguished paper award in the information theory community.
This award is granted to one or two of the most significant information
theory papers worldwide by IEEE the Information Theory Society).

\begin{itemize}
\item This paper addresses the second order analysis of classical channel coding using the information spectrum method,
and solves several {\bf open problems that had been unsolved for 47 years}.

%This paper has been cited 35 times (Google Scholar).
\end{itemize}
\vspace{3 mm}


\bibitem[\textbf{Highly influential papers}]{2}


\bibitem{HN03b} \textbf{M. Hayashi} and H. Nagaoka, 
``General formulas for capacity of classical-quantum channels,'' 
{\em IEEE Transactions on Information Theory,}
Vol. 49, No. 7, 1753--1768 (2003). 
(Citation: \textbf{\textit{130}}. An extremely useful byproduct of this paper is an operator inequality that is widely referred to as the \textit{Hayashi-Nagaoka operator inequality} in literature).

\begin{itemize}
\item This paper studies the quantum channel coding using information spectrum method, a fundamental tool for finite-block-size analysis, and introduces a powerful inequality in quantum Shannon theory.
% as Hayashi-Nagaoka inequality.

\end{itemize}
\vspace{3 mm}


\bibitem{HMMOV03} \textbf{M. Hayashi}, D. Markham, M. Murao, M. Owari, and S. Virmani, 
``Bounds on multipartite entangled orthogonal state discrimination using local
operations and classical communication'' 
{\em Physical Review Letters}, Vol. 96, 040501 (2006). 
(Citation: \textbf{\textit{118}}).

\begin{itemize}
\item This paper introduces a fundamental tool for local distinguishability of quantum states that has been applied in many papers.
It also clarifies the relation between several entanglement quantities.

\end{itemize}
\vspace{3 mm}




\bibitem[\textbf{Resolution of an open problem}]{4}

\bibitem{Ha03b} \textbf{M. Hayashi}, 
``General non-asymptotic and asymptotic formulas in
channel resolvability and identification capacity and its application to wire-tap channel,'' 
{\em IEEE Transactions on Information Theory}, Vol. 52, No. 4, 1562--1575 (2006). 
(Citation: \textbf{\textit{68}}).
 
\begin{itemize}
\item This paper establishes a novel connection between channel resolvability and wiretap channel 
that yields tighter security evaluation, and solves an open problem concerning channel resolvability proposed by Han-Verd\'{u} in 1993.

\end{itemize}
\vspace{3 mm}

\bibitem[\textbf{Papers in quantum information theory}]{5}

\bibitem{NH07b} H. Nagaoka and \textbf{M. Hayashi}, 
``An information-spectrum approach to classical and quantum hypothesis testing for simple hypotheses,''
{\em IEEE Transactions on Information Theory}, 
Vol. 53, No. 2, 534--549 (2007).
(Citation: \textbf{\textit{54}}).


\begin{itemize}
\item 
This paper establishes the quantum information spectrum to address quantum hypothesis testing, which is the most fundamental topic in quantum information and 
provides a basic tool for many related topics.

\end{itemize}
\vspace{3 mm}


\bibitem{aa2}
\textbf{M. Hayashi}, ``Upper bounds of eavesdropper's performances in finite-length code with the decoy method," 
{\em Physical Review A}, Vol.76, 012329 (2007); 
{\em Physical Review A}, Vol.79, 019901(E) (2009).
(Citation: \textbf{\textit{73}}).

\begin{itemize}
\item 
This paper derives the relation between phase error probability and 
leaked information and 
provides a formula for leaked information with imperfect photon source
in quantum key distribution.

\end{itemize}
\vspace{3 mm}

\bibitem[\textbf{Papers in classical information theory}]{6}

\bibitem{Ha08b-1} \textbf{M. Hayashi}, 
``Second-order asymptotics in fixed-length source coding and intrinsic randomness,'' 
{\em IEEE Transactions on Information Theory}, 
Vol. 54, No. 10, 4619--4637 (2008). 
(Citation: \textbf{\textit{62}}).

\begin{itemize}
\item 
This paper proposes the use of the information spectrum method for 
a unified approach for second order asymptotics,
which has been applied in many papers as a key idea.
% for finite-length analysis.

\end{itemize}
\vspace{3 mm}

\bibitem{Ha11b-3} \textbf{M. Hayashi}, 
``Exponential decreasing rate of leaked information in universal random privacy amplification,''
{\em IEEE Transactions on Information Theory}, 
Vol. 57, No. 6, 3989--4001 (2011).
(Citation: \textbf{\textit{48}}).

\begin{itemize}
\item This paper addresses security analysis when hash functions are applied.
It applies hash function to wiretap channel,
and constructs a practical code with small encoding and decoding time.
 
\end{itemize}
\vspace{3 mm}

\bibitem[\textbf{Highly influential review paper}]{3}

\bibitem{WHTH07} 
X.-B. Wang, T. Hiroshima, A. Tomita, and \textbf{M. Hayashi}, 
``Quantum information with Gaussian states,'' 
{\em Physics Reports}, Vol. 448, 1--111 (2007). 
(Citation: \textbf{\textit{131}}).

\begin{itemize}
%\item This paper treats a review of quantum information with Gaussian state including coherent state and squeezed state.

\item This review includes many advanced topics in quantum information based on Gaussian states and is usually referred to
as one of the benchmark reviews for quantum information with Gaussian states.
%such as quantum teleportation, superdense coding, quantum error-correction, quantum cloning, entanglement properties, estimation theory, and quantum cryptography.

\end{itemize}
\vspace{3 mm}

\bibitem[\textbf{Highly influential monograph}]{7}

\bibitem{book-best} \textbf{M. Hayashi}, 
{\em Quantum Information: An Introduction}, 
Springer, 426 pages (2006).
(Citation: \textbf{\textit{243}}.
The publication proposal for the second edition was 
invited by Springer on 31 January 2012
and approved by Springer on 12 March 2012)

\begin{itemize}
\item 
Assuming only elementary knowledge,
this monograph explains many advanced topics 
in quantum information, 
quantum channel coding, quantum data compression, and quantum entanglement, etc.,
including \textbf{original achievements}.

%The new version will be released, including my recent achievements.
%own original research works. 
% with many new advanced topics, 
%including my own original research works. 

%\item This paper has been cited {\bf 190} times (Google Scholar).
\end{itemize}
\vspace{3 mm}


%NJP Impact Factor (2011):4.177
%CMP Impact Factor (2011):1.941
%PRA Impact Factor (2011):2.878

\end{thebibliography}
\end{document}

\newpage
\textbf{A statement on your most significant contributions to the research field of this Proposal.}
%Please attach a PDF of your most significant contributions to the research field of this Proposal (3 pages maximum). Refer to the Instructions to %Applicants for further information. (This question must be answered)

%Upload a PDF of no more than three A4 pages describing your most significant contributions to the research field of this Proposal.
%Describe how your research has led to a significant change or advance of knowledge in your field, and outline how your achievements will contribute %to this Proposal.

%\textbf{\color{red}TO BE EXTENDED!!!}

I have worked in the field of quantum information theory since 1994, and have made a number of significant fundamental and methodological contributions to the areas of 
(1) \textit{information spectrum and finite-block-size analysis}, 
(2) \textit{analysis of practical quantum key distribution systems}, 
and (3) \textit{quantum universal protocols with group symmetries}.
%(4) \textit{information theoretic security},
%and 
%(5) \textit{characterization of multi-partite quantum entanglement}. 
Details follow below:

\textbf{(1) Information spectrum and finite-block-size analysis in classical and quantum information}

Consider the problem of transmitting classical information via classical noisy channels 
as a typical problem in information theory.
This problem is called classical channel coding, and 
it is possible to describe the optimal performance of channel coding 
in the asymptotic limit concerning the block-size of codes as Shannon's channel-coding theorem,
bur, in any real communication system, we can only use codes with finite block-size. 
Hence, 
it is necessary to estimate the optimal coding length $L_{\epsilon,n}$ among codes with block-size $n$
for a given allowable error probability $\epsilon>0$.
In fact, the traditional asymptotic analysis does not work for this problem for the following reason.
In Shannon's channel-coding theorem,
the asymptotic limit $\lim_{n \to \infty} L_{\epsilon,n}/n$ is treated as the channel capacity.
However, the limit does not give a good approximation of $L_{\epsilon,n}$ %with block-size $n$ and $\epsilon>0$
because the convergence is quite slow. 
The reason for the inadequacy of the convergence is that the limit value does not reflect the error probability $\epsilon$.
%not uniform with respect to $\epsilon$, which is caused by the independence of $\epsilon$.
Therefore, a good approximation for $L_{\epsilon,n}$ is required.
This problem becomes more serious in quantum communication
because the quantum key distribution (QKD) system works only with finite-block-size codes.
We cannot apply the result from the asymptotic theory directly to implemented QKD systems. 


To derive a better approximation,
I considered the asymptotic expansion of $L_{\epsilon,n}$ up to the second order $\sqrt{n}$
as $L_{\epsilon,n}\cong L_1 n+ L_{2,\epsilon}\sqrt{n}$
because the second order coefficient $L_{2,\epsilon}$ properly reflects the error probability $\epsilon$, 
so that 
the difference between $L_{\epsilon,n}$ and $L_1 n+ L_{2,\epsilon}\sqrt{n}$ is not so large.
Employing the information spectrum method, I derived the second order coefficient for 
classical channel coding as well as other many kinds of settings
which contain source coding, channel coding, and secure random number extraction 
[IEEE TIT \textbf{54} 4619 (2008), \cite{Ha08b-1} of \textbf{C3}]. %41
%Those results directly give the approximation of the optimal coding length with block-size $n$.
Indeed, although the existing result could solve only the discrete memoryless case for channel coding, 
my result applied to classical channel coding covers 
the Gaussian channel case, the energy constraint case and the Markovian case 
[IEEE TIT \textbf{55} 4947 (2009), \cite{Ha09b} of \textbf{C3}] %41
as well as the discrete memoryless case. 
The information spectrum method addresses only the logarithmic likelihood (ratio) for channels or sources, 
and does not consider the other properties of channels or sources. 
Since the analysis by this method is concentrated on the treatment of the logarithmic likelihood (ratio),
the method provides a unified viewpoint on several topics in information theory.
Thanks to the generality of the method, 
I succeeded in calculating the second order coefficients of the above problems.
%The simplicity of the derivation is clear because of the shortness of the total page lengths and the wideness of areas covered by two papers. 
%In particular, 
The paper [IEEE TIT \textbf{55} 4947 (2009), \cite{Ha09b} of \textbf{C3}] was awarded 
the 
\textbf{\textit{2011 IEEE Information Theory Society Paper Award}}.
This award is presented to one or two of the most significant information theory papers worldwide by the IEEE Information Theory Society. 
Papers awarded this prize are among the most significant and ground-breaking in their year.

Indeed, the topic itself concerns the classical case. 
However, as mentioned in the first paragraph, the problem becomes more serious in the quantum case.
To resolve this problem in the quantum case, I first solved this problem in the classical case.
To address the quantum case,
I established the information spectrum method in quantum systems 
[IEEE TIT \textbf{49} 1753 (2003), \cite{HN03b} of \textbf{C3}] %113
%IEEE TIT \textbf{52} 1562 (2006), %13
%\& IEEE TIT \textbf{53} 534 (2007)] %49
%[\textbf{M. Hayashi}, {\em Quantum Information: An Introduction}, Springer (2006)] 194
with Nagaoka.
%In particular, 
We derived a novel matrix inequality for this purpose, 
which has often been referred to as the
``Hayashi-Nagaoka inequality'' in many papers for quantum channel coding.
We then applied it to the quantum setting and solved the problem of the second order coefficients 
in several quantum cases [QIP 2013, \cite{THc13-1} of \textbf{C2}].%4

The next problem was to derive the upper and lower bounds of the optimal coding length with block-size $n$ 
that attains the above optimal second order coefficient in the asymptotic sense. 
We can easily derive such upper and lower bounds if we are not concerned about the complexity.
However, the computability of bounds is required, as well as the tightness of bounds.
The essential point is the trade-off between the approximation error and the complexity. 
To clarify the trade-off, 
we proposed several upper and lower bounds, and demonstrated their hierarchy. 
%That is, more precise bound requires more calculation time.
Employing the hierarchy, we clarified which bound is most 
useful when the block-size $n$ is sufficiently large [QIP 2013, \cite{THc13-1} of \textbf{C2}].


According to Google Scholar, this research has been cited \textbf{\textit{455}} %41+41+113+13+49+194+4
 times since 2003.

\textbf{(2) Analysis of practical quantum key distribution systems}

Quantum key distribution (QKD) is the theoretical proposal that is close to realisation in quantum information science. 
While an ideal QKD system is secure, 
the security of a realisable QKD system requires additional protocols and more careful discussion 
because of the many kinds of imperfections, e.g., noise of the quantum channel and imperfection of photon source. 
%Due to these imperfections, 
We need to attach error correction and privacy amplification to the conventional QKD protocol. Further, imperfection in photon sources introduces additional channel parameters that should be estimated.

Based on my work on wire-tap channels 
[IEEE TIT \textbf{52} 1562 (2006), \cite{Ha03b} of \textbf{C3}], %50
% IEEE TIT \textbf{57} 3989 (2011)]. %29
I developed a concrete and secure QKD protocol with small calculation time
while existing researches offered no such concrete protocol
[PRA \textbf{76} 012329 (2007)]. %57
Many implemented QKD systems utilise this protocol. 
The first essential point was to propose a protocol 
for privacy amplification by modifying a Toeplitz matrix whose calculation time is linear in the block size $n$ 
[PRA \textbf{76} 012329 (2007)]. 
Second, I proposed a method to perfectly estimate the additional channel parameters
whereas existing methods only estimated it partially 
[NJP \textbf{9} 284 (2007) \& \cite{Hp8} of \textbf{C2}]. %20
As a result of this improvement, the key generation rate was extensively improved. 
Third, I clarified the amount of leaked information and derived the sacrificed bit length required for an arbitrary desired security 
level for the estimated channel parameters in 
the finite-block-size setting 
[PRA \textbf{74} 022307 (2006) %39
\& PRA \textbf{76} 012329 (2007)] by using the trade-off between accessible information and quantum disturbance 
[PRL \textbf{100} 210504 (2008), \cite{BHH08} of \textbf{C3} ]. %23
Finally, in cooperation with the above channel parameter estimation, I derived a calculation formula for the sacrificed bit length with the form applicable to the realised QKD system 
[NJP \textbf{14} 093014 (2012), \cite{HT12} of \textbf{C2} \& \cite{Hp8} of \textbf{C3}]. %3
%in preparation]. 
Using this formula, I addressed the trade-off between cost and accuracy for the estimation of leaked information 
[PRA \textbf{79} 020303(R) (2009), \cite{Ha09-1} of \textbf{C2}]. %4
This formula was chosen for use in the implemented QKD system by this project in Japan, 
which will be demonstrated over an installed optical fibre
because my analysis satisfies all practical and experimental requirements for the NICT project.

According to Google Scholar, this research has been cited \textbf{\textit{225}}
 %50+29+57+20+39+23+3+4
 times since 2006.

\renewcommand{\refname}{}
\vspace{-20 mm}
\begin{thebibliography}{99}
\bibitem[\textbf{References}]{r0}

\bibitem[PRA \textbf{76} 012329 (2007)]{r1} 
\textbf{M. Hayashi},
%\textbf{M. Hayashi}, 
``Upper bounds of eavesdropper's performances in finite-block-size coding with the decoy method,'' 
{\em Physical Review A}, Vol. 76, 012329 (2007).

\bibitem[NJP \textbf{9} 284 (2007)]{r2}
\textbf{M. Hayashi},
%\textbf{M. Hayashi}, 
``General theory for decoy-state quantum key distribution with an arbitrary number of intensities,'' 
{\em New Journal of Physics}, Vol. 9, 284 (2007).

\bibitem[PRA \textbf{74} 022307 (2006)]{r3}
\textbf{M. Hayashi},
%\textbf{M. Hayashi}, 
``Practical evaluation of security for quantum key distribution,'' 
{\em Physical Review A}, Vol. 74, 022307 (2006).

\end{thebibliography}

\textbf{(3) Quantum universal protocols with group symmetries}

Quantum information has several information protocols, e.g., quantum source coding, quantum channel coding, and perfect entangled state generation from partial entangled states. 
Existing protocols depend on the parameters that describe the quantum information source and the quantum channel. 
However, it is quite difficult to identify these parameters in a quantum system.
Even if it is possible, the identification of the parameters may cause information loss due to the state reduction by measurement. 
Hence, protocols are required that universally work independently of these parameters. 
Using the method of types, 
Csiszar and K\"{o}rner [\emph{Information Theory: Coding Theorems for Discrete Memoryless Systems}, Akad\'emiai Kiad\'o (1981)]
proposed universal protocols for classical source coding and classical channel coding. 
By contrast, in quantum systems, quantum universal protocols are required to have basis independence.
Hence, we cannot directly apply the original type method to quantum systems because 
the original type method can work only with an specific basis.
%In quantum systems, the universal protocols are required to works independently of the coordinate.
%However, to apply the original type method to quantum systems directly, we have to choose a basis.

%To resolve this problem, I remembered that
%the group representation theory enables us to treat physical properties independent of the coordinate.
I showed that the group representation theory can greatly simplify 
quantum state estimation with the $n$-copy case
by removing the basis dependence
[JPA \textbf{31} 4633 (1998)]. %27
%\& JPA \textbf{33} 7793 (2000)]. %9
%Physics Reports, \textbf{448} 1 %107 
%Proc. QCMC 2, 99 (2002) %7
%JMP 49 102101 (2008) %33
%JPA 35 7689 (2002) %30   
I then found a quite unexpected and remarkable similarity between the Schur duality in the group representation theory and the method of types
%[JPA \textbf{34} 3413 (2001) %19 \& 
[JPA \textbf{35} 10759 (2002)]. %38
Based on this discovery, I invented quantum universal protocols for all of the above quantum coding problems 
%[PRA \textbf{66} 022311 (2002), %41
%PRA \textbf{66} 032321 (2002), %15
%QIC \textbf{2} 519 (2002), %16
%PRA \textbf{75} 062338 (2007), %16
[CMP \textbf{289} 1087 (2009), \cite{Ha09-4} of \textbf{C2} \& %9
CMP \textbf{293} 171 (2010), \cite{Ha09-7} of \textbf{C2}]. %5
Thanks to these protocols, we can now choose our protocol without knowledge of the parameters of 
quantum information sources or channels. 
This property enables us to avoid additional measurements to identify the quantum system. 
In particular, using irreducible representation theory, I realised basis independence for the above protocols. 
I also guaranteed their performance by the sophisticated use of matrix inequalities.

I subsequently applied these techniques 
to the problem of distinguishing multi-partite entangled states [PRL \textbf{96} 040501 (2006), \cite{HMMOV03} of \textbf{C3}]. %102
%PRA \textbf{77} 012104 (2008) %55 
%JMP \textbf{50} 122104 (2009) %25
%NJP \textbf{12} 083002 (2010)]. %13
Since multi-partite entanglement is related to multi-user communication, in particular, to quantum networks 
[PRA \textbf{76} 040301(R) (2007)], % 17
%LNCS \textbf{4393} 610 (2007)], %26
these results are helpful for this project. 

According to Google Scholar, this research has been cited \textbf{\textit{610}}  
%27+9+107+7+33+30+19+38+41+15+16+16+9+5+102+55+25+13+17+26
times since 1998.

\renewcommand{\refname}{}
\vspace{-20 mm}
\begin{thebibliography}{99}
\bibitem[\textbf{References}]{s0}

\bibitem[JPA \textbf{31} 4633 (1998)]{s1} 
\textbf{M. Hayashi},
%\textbf{M. Hayashi}, 
``Asymptotic estimation theory for a finite dimensional pure state model,'' 
{\em Journal of Physics A: Mathematical and General}, Vol. 31, No. 20, 4633-4655 (1998).

\bibitem[JPA \textbf{35} 10759 (2002)]{s2}
\textbf{M. Hayashi},
%\textbf{M. Hayashi}, 
``Optimal sequence of quantum measurements in the sense of Stein's lemma in quantum hypothesis testing,'' 
{\em Journal of Physics A: Mathematical and General}, Vol. 35, No. 50, 10759-10773 (2002).

\bibitem[PRA \textbf{76} 040301(R) (2007)]{s3}
\textbf{M. Hayashi},
%\textbf{M. Hayashi}, 
``Prior entanglement between senders enables perfect quantum network coding with modification,'' 
{\em Physical Review A}, Vol. 76, 040301(R) (2007).

\end{thebibliography}


